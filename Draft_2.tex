\documentclass[letter]{article}

\usepackage{fullpage} % Package to use full page
%\usepackage{parskip} % Package to tweak paragraph skipping
\usepackage{tikz} % Package for drawing
\usepackage{amsmath}
\usepackage{hyperref}
\usepackage{mathtools}
\usepackage{listings}

\usepackage{caption} 
\captionsetup[table]{skip=10pt}

\title{TEM Waves Draft}
\author{Panya Sukphranee}
\date{\today}

\begin{document}

\maketitle

\section{Draft}
	
	\indent We look for a numerical solution for \\
	\begin{align*}
		\vec{E} &=  E_0 U(x, y) e^{i(\omega t - \beta z)} \hat{y}
	\end{align*}
	in a waveguide with varying refractive indices along the $\hat{x}$ and $\hat{y}$ axes.
	\begin{align*}
		\frac{\partial^2 U}{\partial x^2} + \frac{\partial^2 U}{\partial y^2} - \beta^2 U &= - \mu \epsilon \omega^2 U \\
		\frac{\partial^2 U}{\partial x^2} + \frac{\partial^2 U}{\partial y^2} + \mu \epsilon \omega^2 U &= \beta^2 U \\
		\frac{\partial^2 U}{\partial x^2} + \frac{\partial^2 U}{\partial y^2} + n(x,y)^2 k_0^2 U &= \beta^2 U \\
	\end{align*}		
	Making the above discrete gives us\\
	\begin{align*}
		\frac{U_{i+1, j} + U_{i-1,j} - 2U_{i,j}}{\Delta x^2} + \frac{U_{i, j+1} + U_{i,j-1} - 2U_{i,j}}{\Delta y^2}
			+ {n_{i,j}}^2 k_0^2 U_{i,j} &= \beta^2 U_{i,j} \\
	\end{align*}
	$\Delta y \rightarrow \Delta x$
	\begin{align*}
		U_{i+1, j} + U_{i-1,j} - 2U_{i,j} + U_{i, j+1} + U_{i,j-1} - 2U_{i,j}
			+ {n_{i,j}}^2 {\Delta x}^2 k_0^2 U_{i,j} &= \beta^2 {\Delta x}^2 U_{i,j} \\
		U_{i+1, j} + U_{i-1,j} - 4U_{i,j} + {n_{i,j}}^2 {\Delta x}^2 k_0^2 U_{i,j} + U_{i, j+1} + U_{i,j-1} 
					 &= \beta^2 {\Delta x}^2 U_{i,j} \\
		U_{i+1, j} + U_{i-1,j} + ({n_{i,j}}^2 {\Delta x}^2 k_0^2 - 4)U_{i,j} + U_{i, j+1} + U_{i,j-1} 
					 &= \beta^2 {\Delta x}^2 U_{i,j} \\
	\end{align*}
	Let $\alpha_{i,j} =({n_{i,j}}^2 {\Delta x}^2 k_0^2 - 4)$\\
	\begin{align}
		\label{eqn:U}
		U_{i+1, j} + U_{i-1,j} + \alpha_{i,j}U_{i,j} + U_{i, j+1} + U_{i,j-1} 
					 &= \beta^2 {\Delta x}^2 U_{i,j}
	\end{align}
	$U$ is an $m \times n$ matrix. $m$ rows, $n$ columns.\\
	$U_{i,j}$ is the $i^{th}$ row, $j^{th}$ column. $(1 \leq i \leq m), (1 \leq j \leq n)$.\\
	We want to rewrite matrix $U$ as a column vector $V$.
	\begin{figure}	
	\[ 		
	\mathbf{U} = 
	\begin{bmatrix}
		a_{11}	 	& a_{12}	& a_{13} 	& \hdots	& 	a_{1n}	 	\\
		a_{21} 		& a_{22}	& 	 		& 			& 	\vdots	 	\\
		a_{31}		& 			& 	 		& 			&	\vdots		\\
		\vdots		& 			& 			& \ddots	&	\vdots		\\
		a_{m1}	 	& a_{m2}	& \hdots	& \hdots	& 	a_{mn}
	\end{bmatrix}
	\Rightarrow
	\begin{bmatrix}
		a_{11}\\
		a_{21}\\
		\vdots\\
		a_{m1}\\
		a_{12}\\
		\vdots\\
		a_{m2}\\
		\vdots\\
		a_{mn}
	\end{bmatrix}
	= \mathbf{V}		
	\]
	\caption{Mapping U to V}
	\label{matrix:U-V}	
	\end{figure}
	$U_{l,k} \mapsto V_{m(k-1)+l}$*.\\
	The map is 1-1 because,
	\begin{align*}
		\text{Suppose } m(k' - 1) + l' &= m(k-1)+l,\\
		\Rightarrow	m(k' - k) &= l - l',\\
		\\
		\text{then } |m(k'-k)| &\leq m-1, \text{ since $|l - l'| \leq m-1$},\\
		|k' - k| &\leq  \frac{1}{m} - 1 \\
		\Rightarrow l=l' &\Rightarrow k=k'. 
	\end{align*}
	\textbf{Note that indices i and j of U both start at 1 impies the index of V starts at 1.}\\
	Below is a tabulation of indices appearing in equation $\ref{eqn:U}$ to see what index of $V$ it maps to.\\
	\begin{table}[h]
	\centering
	\caption{Index Table}
	\label{index_table}
	\begin{tabular}{llll}
	\multicolumn{1}{c}{k} & \multicolumn{1}{c}{l} & \multicolumn{1}{c}{m(l-1) + k}      & \multicolumn{1}{c}{} \\ \cline{1-3}
	i + 1                 & j                     & \multicolumn{1}{l|}{mi + j}			 	& m(j-1) + i + 1       \\
	i - 1                 & j                     & \multicolumn{1}{l|}{m(i-2) + j} 		& m(j-1) + i - 1       \\
	i                     & j                  	  & \multicolumn{1}{l|}{m(i-1) + j}     	& m(j-1) + i          \\
	i                     & j+1                   & \multicolumn{1}{l|}{m(i-1) +j + 1}      & m(j-1) + i + m\\
	i                     & j-1                   & \multicolumn{1}{l|}{m(i-1) + j - 1}     & m(j-1) + i - m  \\
	\end{tabular}
	\end{table}
	
	Equation $\ref{eqn:U}$ becomes\\
	\begin{align*}
		V_{m(j-1) + i + 1} + V_{ m(j-1) + i - 1} + \alpha_{i,j} V_{m(j-1) + i} + V_{m(j-1) + i + m} + V_{m(j-1) + i - m} &= \beta^2 {\Delta x}^2 V_{m(j-1) + i }\\
	\end{align*}
	rewrite in increasing index order,\\
	\begin{align}
		\label{eqn:V}
		V_{m(j-1) + i -m} + V_{ m(j-1) + i - 1} + \alpha_{i,j} V_{m(j-1) + i} + V_{m(j-1) + i + 1} + V_{m(j-1) + i + m} &= \beta^2 {\Delta x}^2 V_{m(j-1) + i }
	\end{align}
	
	The index on the first and last terms tells us that this relationship isn't valid for every entry $V_k$. So we must find the upper and lower indices for which this relationship is valid; and manually find the other entries. The lowest index has to be $\geq 1$. The highest index $\leq mn$. By substituting $p=m(j-1) + i$,\\
	\begin{align}
		V_{p - m} + V_{p - 1} + \alpha_{i,j} V_{p} + V_{p + 1} + V_{p + m} &= \beta^2 {\Delta x}^2 V_{p}
		\label{eqn:V_p}
	\end{align}
	we can see that $m+1 \leq p \leq m(n-1)$.\\
	
	Now we manipulate inequalities to find out which (i,j)'s correspond to these limits.
	Note that $k = k(i,j) = m(j-1) + i$ is also an increasing function of both i and j, (i.e. $\frac{\partial k}{\partial i}$,$\frac{\partial k}{\partial j}>0$), so after finding (i,j)'s satisfying the above, we will know which other one's to exclude. Equations $\ref{eqn:lower_limit}$ and $\ref{eqn:upper_limit}$ shows that ($\ref{eqn:V_p}$) applies to the inner columns of matrix U. \\
	\begin{equation} \label{eqn:lower_limit}
		\begin{split}
		m(j-1) + i - m &\geq 1\\
		m(j-2) + i &\geq 1\\
		\Aboxed{i = 1, j &= 2}\\
		\end{split}
	\end{equation} 

	\begin{equation} \label{eqn:upper_limit}
		\begin{split}
		m(j-1) + i + m &\leq mn\\
		m(j -n) + i &\leq 0\\
		\Aboxed{i=m, j&=n-1}\\
		\end{split}
	\end{equation}
	
	The outer column entries (of matrix U) equal zero because they are boundaries. We now have an eigenvalue problem of an $m(n-2) \times m(n-2)$ matrix. Equation $\ref{eqn:V_p}$ gives the following matrix. \\
	
	\begin{figure}[h]	
	\[ 		 
	\begin{bmatrix}
		\alpha_{21} & 1              & 0              & 0              & \hdots & \hdots & 1    & \hdots         & 0              \\
1            & \alpha_{31} & 1              & 0              & 0      &        &                &                & 0              \\
0            & 1              & \alpha_{41} & 1              & 0      & 0      &                &                & 0              \\
0            & 0              & 1              & \alpha_{51} & 1      & 0      & 0              &                & 0              \\
\vdots       &  0      &                &                & \ddots &        &                &                & \vdots         \\
\vdots       &  \vdots &                &                &        & \ddots &                &                & \vdots         \\
\vdots       &  \vdots &                &                &        & \hdots      & \alpha_{n-3,m} & 1              & 0              \\
\vdots       &  \vdots &                &                &        & 0      & 1              & \alpha_{n-2,m} & 1              \\
0            & \hdots         & \hdots         & \hdots         & \hdots & \hdots & 0              & 1              & \alpha_{n-1,m}
	\end{bmatrix}
	\begin{bmatrix}
		V_{m+1}\\
		V_{m+2}\\
		\vdots\\
		\vdots\\
		\vdots\\
		\vdots\\
		\vdots\\
		\vdots\\
		\vdots\\
		\vdots\\
		V_{m(n-1)}
	\end{bmatrix}
	= \beta^2 {\Delta x}^2
	\begin{bmatrix}
		V_{m+1}\\
		V_{m+2}\\
		\vdots\\
		\vdots\\
		\vdots\\
		\vdots\\
		\vdots\\
		\vdots\\
		\vdots\\
		\vdots\\
		V_{m(n-1)}
	\end{bmatrix}		
	\]
	\caption{Eigen Matrix}
	\label{matrix:eigenMatrix}	
	\end{figure}
	
	
	
	
\end{document}
              